%!TEX root = main.tex

\section{Track Record}

\subsection{Prof. Steven Murdoch}



\subsection{Dr Aydin Abadi}


He is a Senior Research Fellow at UCL’s Department of Computer Science (2021--present). Before holding that position, he held a Lectureship position at the University of Gloucestershire (2020--2021) and before that he was a Research Associate at the Blockchain Technology Lab, at the University of Edinburgh (2017--2020). Currently, he is a member of the ``National Research Centre on Privacy, Harm Reduction and Adversarial Influence Online'', a.k.a. REPHRAIN, (2021--present), and an associate member of the Blockchain Technology Lab, at the University of Edinburgh (2020--present).

His primary research interests include cryptography and cryptocurrency, with a focus on (a) payment fraud, (b) developing Privacy Enhancing Technologies (PETs), such as Private Set Intersection (PSI), time-lock encryptions, and zero-knowledge proofs, (c) blockchain technology, and (d) cloud computing. Since 2017, when he received his PhD (in cryptography and PSI), he has been leading the development and implementation of various cryptographic protocols. 

The privacy-preserving techniques he pioneered have become staples in the field of PSI, see \href{https://link.springer.com/chapter/10.1007/978-3-319-18467-8_1}{[\printcntr]} and \href{https://link.springer.com/chapter/10.1007/978-3-662-54970-4_9}{[\printcntr]}. He designed the first secure delegated PSI that lets data storage and private computation be outsourced to powerful but potentially malicious cloud computing. He also improved the run-time of state-of-the-art delegated PSI protocols by orders of magnitude, see \href{https://ieeexplore.ieee.org/document/7934388}{[\printcntr]}. To date, the updatable PSI that he discovered remains the only PSI that can support data updates with very low communication and computation costs \href{https://link.springer.com/chapter/10.1007/978-3-031-18283-9_6}{[\printcntr]}.  


He has also contributed to the security of online banking; specifically, he designed the first security protocol that helps honest victims of online banking fraud prove their innocence, and receive compensation for their financial losses, see \href{https://eprint.iacr.org/2022/107.pdf}{[\printcntr]}. He has several publications in the field of blockchain technology and cryptocurrency as well; for instance, he designed a generic fair exchange protocol that allows users to securely pay in cryptocurrency for any verifiable digital services if and only if they receive the promised services; his proposed protocol can preserve users' privacy and can prevent fraud in the case where either a malicious service provider wants to receive payment without delivering the service or a malicious user wants to use a digital service without paying the service provider, see \href{https://arxiv.org/pdf/2208.00283.pdf}{[\printcntr]}. He has developed two decentralised platforms that are still functional and online (see \href{http://blockchainlab.inf.ed.ac.uk/id-management/#/}{[\printcntr]} and \href{http://blockchainlab.inf.ed.ac.uk/valued/}{[\printcntr]} for more details).


Very recently, he as part of a team (consisting of Prof. Steven Murdoch, Privitar, and researchers from Cardiff University) won the first phase of the “UK-US Privacy Enhancing Technologies Challenge Prize” which resulted in attracting £60k in funding, see \href{https://www.ucl.ac.uk/computer-science/news/2022/dec/ucl-computer-sciences-success-privacy-enhancing-technologies-challenge}{[\printcntr]}.  He has over 20 technical papers and 6 open-source software/prototypes in the field of cryptography and cryptocurrency. Five of his papers are at CORE “A” ranked conferences and journals.