%!TEX root = main.tex

\



\section{Track Record}

\subsection{Prof. Steven Murdoch}

Steven Murdoch is Professor of Security Engineering at University College London and is the head of the Information Security Research Group. 

Professor Murdoch has worked extensively in payment system security, including fraud prevention and consumer protection.
His proposed security measures have been adopted in the EMV protocol, now the most widely used smart card payment system worldwide.
He also led the commercialisation of a new authentication scheme for online banking that is used by the largest banks in Europe (including Commerzbank, Deutschebank and Rabobank).
This system was acquired by OneSpan, a leading provider of authentication products in the financial industry.

His work on protecting vulnerable banking customers from unfair treatment has guided the development of the current consumer protection scheme against push payment fraud, and he is currently working on how this should be revised based on experience of this being in use.

Professor Murdoch is also closely involved in the relationship between computer science and the law, and is regularly an expert witness in disputes over fraudulent payments.
He is the author of ``The sources and characteristics of electronic evidence and artificial intelligence'' (University of London Press) and is leading the writing team on a Royal Society
project to provide guidance to the judiciary on the interpretation of electronic evidence in court.

In the field of anonymous communications, Professor Murdoch has developed both theoretical and practical breakthroughs in privacy enhancing technologies.
He has applied game-theory to the design of censorship-resistance schemes and the results have been adopted by the Tor Project.
He was also the creator of Tor Browser, the primary means for users to access the Tor anonymous communication network.

\subsection{Dr Aydin Abadi}


Aydin Abadi is a Senior Research Fellow at UCL’s Department of Computer Science.  He also held a Lectureship position at the University of Gloucestershire and before that he was a Research Associate at the Blockchain Technology Lab, at the University of Edinburgh. His primary research interests include cryptography and cryptocurrency, with a focus on (a) payment fraud, (b) developing Privacy Enhancing Technologies (PETs), (c) blockchain technology, and (d) cloud computing. Since 2017, when he received his PhD, in Private Set Intersection (PSI), he has been leading the development and implementation of various cryptographic protocols. 

The privacy-preserving techniques he pioneered have become staples in the field of PSI, see \href{https://link.springer.com/chapter/10.1007/978-3-319-18467-8_1}{[\x]} and \href{https://link.springer.com/chapter/10.1007/978-3-662-54970-4_9}{[\x]}. He designed the first delegated PSI that lets data storage and private computation be outsourced to powerful but potentially malicious cloud computing. To date, the updatable PSI that he discovered remains the only PSI supporting data updates with low communication and computation costs \href{https://link.springer.com/chapter/10.1007/978-3-031-18283-9_6}{[\x]}.  


He has also contributed to the (theoretical) security of online banking to help honest victims of online banking fraud prove their innocence, and receive compensation for their financial losses, see \href{https://eprint.iacr.org/2022/107.pdf}{[\x]}. He has several publications in the field of blockchain technology and cryptocurrency as well; for instance, he designed a generic fair exchange protocol that lets users securely pay in cryptocurrency for any verifiable digital services if and only if they receive the promised services; his proposed protocol can preserve users' privacy and can prevent fraud in the case where either a malicious service provider wants to receive payment without delivering the service, see \href{https://arxiv.org/pdf/2208.00283.pdf}{[\x]}. He has developed two decentralised (i.e., blockchain-based) platforms that are still functional and online (see \href{http://blockchainlab.inf.ed.ac.uk/id-management/#/}{[\x]} and \href{http://blockchainlab.inf.ed.ac.uk/valued/}{[\x]} for more details).


Recently, he as part of a team (consisting of Prof. Steven Murdoch and a company called Privitar) won the first phase of the “UK-US Privacy Enhancing Technologies Challenge Prize” resulting in attracting £60k in funding, see \href{https://www.ucl.ac.uk/computer-science/news/2022/dec/ucl-computer-sciences-success-privacy-enhancing-technologies-challenge}{[\x]}.  He has over 20 technical papers and 6 open-source prototypes in the field of cryptography and cryptocurrency. Five of his papers are at CORE “A” ranked conferences and journals.


%!TEX root = main.tex





\section*{References}
%
\textbf{[\y]} Aydin Abadi, Sotirios Terzis, Changyu Dong, O-PSI: Delegated Private Set Intersection on Outsourced Datasets, 2015, \textbf{[\y]} Aydin Abadi, Sotirios Terzis, Changyu Dong, VD-PSI: Verifiable Delegated Private Set Intersection on Outsourced Private Datasets, 2017, \textbf{[\y]} Aydin Abadi, Changyu Dong, Steven Murdoch, Sotirios Terzis, Multi-party Updatable Delegated Private Set Intersection, 2022, \textbf{[\y]} Aydin Abadi, Steven Murdoch, Payment with Dispute Resolution:
A Protocol For Reimbursing Frauds Victims, 2022, \textbf{[\y]} Aydin Abadi, Steven Murdoch, Thomas Zacharias, Recurring Contingent Service Payment, 2022, \textbf{[\y]} Aydin Abadi, Aggelos Kiayias, Lamprini Georgiou, Jin Xiao, Dave Cochran, Privacy-preserving Identity Management System, 2020, http://blockchainlab.inf.ed.ac.uk/id-management, \textbf{[\y]} Aydin Abadi, Richard Shillcock, Dave Cochran, 2019, http://blockchainlab.inf.ed.ac.uk/valued, \textbf{[\y]} UCL Computer Science, UCL Computer Science's success in Privacy-Enhancing Technologies' challenge, 2022, \href{https://www.ucl.ac.uk/computer-science/news/2022/dec/ucl-computer-sciences-success-privacy-enhancing-technologies-challenge}{https://www.ucl.ac.uk/computer-science/news/2022/dec/ucl-computer-sciences-success-privacy-enhancing-technologies-challenge}. 





%S. Venkataramakrishnan, J. Oliver, UK unveils bid to become global hub for crypto, 2022, \textbf{[2]} Action Fraud, Cryptocurrency fraud leads to millions in losses so far this year, 2021, \textbf{[3]} Santander, Santander warns about celebrity endorsed crypto scams, 2022, \textbf{[4]} E. Fletcher, Reports show scammers cashing in on crypto craze, 2022
%\textbf{[5]} D. Gilbert, Inside the QAnon Crypto Scam That Cost People Millions and One Man His Life, 2022 
%\textbf{[6]} G. Olukya, The billion-dollar cryptocurrency scams you’ve never heard about, 2022,
%\textbf{[7]} Arab News, Thousands fall victim to \$2bn Turkish cryptocurrency fraud, 2021
%\textbf{[8]} Sakshi Post, Suryapet Man's Suicide Over Cryptocurrency Fraud Could be Tip of the Iceberg , 2021, \textbf{[9]} Reddit, Someone has committed suicide after losing their live savings in the SnowdogDAO rug pull, 2022, \textbf{[10]} F. Toosi, J. Buckley, Using artificial intelligence to detect fraud on the blockchain, 2019, \textbf{[11]} E. Jung, M. Tilly, A. Gehani, Data Mining-based Ethereum Fraud Detection, 2019, \textbf{[12]} M. Li, A Survey on Ethereum Illicit Detection, 2022, \textbf{[13]} M. Rosenthal, Finding Coverage for Cryptocurrency Losses, 2022, \textbf{[14]} D. Zhang, Crypto Risks, Uncertainty Prompt Uptick in Insurance Exclusions, 2022, \textbf{[15]} Y. Lindell, How to Simulate It: A Tutorial on the Simulation Proof Technique, 2017, \textbf{[16]} Aydin Abadi, Steven Murdoch, Payment with Dispute Resolution: A Protocol For Reimbursing Frauds Victims, 2022, \textbf{[17]} Aydin Abadi, Delegated private set intersection on outsourced private datasets, 2017, \textbf{[18]} C. Dong, Y. Wang, Smart Counter-Collusion Contracts for Verifiable Cloud Computing, 2017, \textbf{[19]} H. Shacham, B. Waters, Compact Proofs of Retrievability, 2008, \textbf{[20]} I Damgard, M. Koprowski , Practical Threshold RSA Signatures without a Trusted Dealer, 2001, \textbf{[21]} R. Gennaro, S. Goldfeder, Threshold-Optimal DSA/ECDSA Signatures and an Application to Bitcoin Wallet Security, 2016,  \textbf{[22]} R. Gennaro, C. Gentry, Non-interactive Verifiable Computing: Outsourcing Computation to Untrusted Workers, 2010, \textbf{[23]} Aydin Abadi, Source code of SAP and smart contracts of recurring payments for proofs of retrievability, 2022, https://github.com/AydinAbadi/RC-S-P/tree/main/RC-PoR-P-Source-cod, \textbf{[24]} Aydin Abadi, Source code of client in recurring payments for proofs of retrievability, 2022, https://github.com/AydinAbadi/RC-S-P/blob/main/RC-PoR-P-Source-cod/RC-PoR-P.cpp, \textbf{[25]} Aydin Abadi, Source code of threshold e-voting protocols, 2022, https://github.com/AydinAbadi/PwDR/blob/main/PwDR-code/generic-encoding-decoding.cpp, \textbf{[26]} Aydin Abadi, Steven Murdoch, Thomas Zacharias, Recurring Contingent Service Payment, 2022. 




