%!TEX root = main.tex


\section{Background}\label{sec::intro}
%\vspace{-2mm}

%An  ``Authorised Push Payment" (APP) fraud is a type of cybercrime where fraudsters trick a victim into making an online payment into an account controlled by them. The ``Financial Conduct Authority” (FCA) defines this type of fraud as \textit{``a transfer of funds by person $A$ to person $B$, other than a transfer initiated by or through person $B$, where: (1) $A$ intended to transfer the funds to a person other than $B$ but was instead deceived into transferring the funds to $B$; or (2) A transferred funds to $B$ for what they believed were legitimate purposes but which were, in fact, fraudulent''} [1]. APP fraud has various variants, such as purchase, romance, or (celebrity-endorsed) investment fraud [2].  
%%
%
%According to a report produced by ``UK Finance'' \emph{online} payment is the type of payment method the victims used to make the authorised push payment in  $98\%$  of cases. This type of fraud is affecting the UK's residents and economy. APP fraud can be categorised into two broad classes, Online Banking fraud and cryptocurrency fraud. The latter is the focus of this research.   
%%







 Cryptocurrency has evolved from a niche application developed by activists to a wide-scale form of payment. This trend is likely to accelerate as a result of banks' interests, e.g., Central Bank Digital Currencies (CBDCs), UK government support and plan to become a ``global hub'' for the cryptocurrency industry \href{https://www.ft.com/content/24c9b6de-9cc6-4413-8b6a-e60653a29ce0?shareType=nongift}{[\printcntr]}, and industry initiatives to embed cryptocurrency payments in popular applications, e.g., Revolut.\footnote{{\small{To save space, all references in this document have been hyperlinked.}}} However, cryptocurrencies have also drawn the attention of criminals who want to steal digital currency. According to the UK National Fraud and Cyber Crime Reporting Centre over \textbf{£146 million was lost to cryptocurrency fraud} in 2021 \href{https://www.actionfraud.police.uk/news/cryptocurrency-fraud-leads-to-millions-in-losses-so-far-this-year} {[\printcntr]}, a 30\% increase since 2020. Santander reports an 87\% increase in the volume of cases of this type of fraud in 2022, compared to 2021 \href{https://www.santander.co.uk/about-santander/media-centre/press-releases/santander-warns-about-celebrity-endorsed-crypto-scams}{[\printcntr]}. In the UK, victims in the age range of 18–25 account for the highest percentage of reports related to cryptocurrency fraud, i.e., 11\%, \href{https://www.actionfraud.police.uk/news/cryptocurrency-fraud-leads-to-millions-in-losses-so-far-this-year} {[2]}. The harm resulting from cryptocurrency fraud is not unique to the UK. In the US, the Federal Trade Commission suggested that more than 46,000 people reported losing over \$1 billion in cryptocurrency to fraud, from January 2021 to March 2022 with 34\% of the victims in the age range of 18–29 \href{https://www.ftc.gov/news-events/data-visualizations/data-spotlight/2022/06/reports-show-scammers-cashing-crypto-craze}{[\printcntr]}. The true cost of cryptocurrency fraud often extends beyond the immediate financial loss, including law-enforcement time and dealing with the emotional fallout of the fraud. Unfortunately, there have been reports of \textbf{suicides} of cryptocurrency fraud victims 
 %
 \href{https://www.vice.com/en/article/pkgda8/qanon-crypto-scam-whiplash347}{[\printcntr]},  \href{https://www.ozy.com/around-the-world/the-billion-dollar-crypto-currency-scams-youve-never-heard-about/266860/}{[\printcntr]},  \href{https://www.arabnews.com/node/1847671/middle-east}{[\printcntr]},  \href{https://english.sakshi.com/news/crime/suryapet-mans-suicide-over-cryptocurrency-fraud-could-be-tip-iceberg-147223}{[\printcntr]},  \href{https://www.reddit.com/r/CryptoCurrency/comments/r4asjl/someone_has_committed_suicide_after_losing_their/}{[\printcntr]}. 
 




%\printcntr


%Cryptocurrency has been drawing the attention of individuals in the UK and around the world. It has also drawn considerable attention from fraudsters who want to illegally transfer digital currency from a victim’s account to their account. According to the UK's ``National Fraud and Cyber Crime Reporting Centre'' over £146 million was lost to cryptocurrency fraud in 2021 [3], which is $30\%$ more than the amount, lost to this type of fraud in 2020. According to Santander bank, during the first quarter of 2022, only ``investing in cryptocurrencies endorsed by celebrities'' fraud spiked by $61\%$ compared to the last quarter of 2021. Santander bank estimated an $87\%$ increase in the volume of cases of this type of fraud in 2022, compared to 2021 [4]. \textbf{So far, in the UK, victims in the age range of 18--25 account for the highest percentage of reports related to cryptocurrency fraud}, i.e., 11\%, [3]. 
%
%
%
%
%Cryptocurrency fraud is a global phenomenon. A report produced by the USA's ``Federal Trade Commission'' (FTC) suggests that more than 46,000 people reported losing over \$1 billion in cryptocurrency to fraud, from January 2021 through March 2022 [5]. According to this report, the top cryptocurrencies people said they used to pay fraudsters were Bitcoin (70\%), Tether (10\%), and Ethereum (9\%). Despite the striking amount of money lost to cryptocurrency fraud, there has been no scientific approach to protect cryptocurrency users that have fallen victim to cryptocurrency fraud.  This report suggests that 34\% of the cryptocurrency fraud victims were in the age range of 18--29. According to FTC, younger consumers were more likely to fall victim to cryptocurrency fraud. 
%
%



%, 35% (30-39), 33% (40-49), 28% (50-59), 19% (60-69), 10% (70-79), and 2% (80 and over). These figures exclude reports that did not indicate age.

%In the UK, victims aged under 45 now account for 70\% of reported investment scams, according to new data from Lloyds Bank. 

%Thus, there is a pressing call for effective solutions that can deal with this type of fraud, which has been affecting people in the UK and all around the world. %Such an urgent need has been emphasised by the UK's ``HM Treasury'' very recently, in May 2022 \cite{HM-Treasury-APP-Fraud}. 



%%%%%%%%%%%%%%%%
%Secondly, victims' protection level is low. In the first half of 2021, only $42\%$ of the stolen funds returned to victims of  APP frauds in the UK \cite{2021-Half-Year-Fraud-Update}. Previous years had even lower rates than that.  Despite the UK's financial regulators (unlike US and EU) having provided specific reimbursement regulations (e.g., CRM code) to financial institutes to improve APP fraud victims' protection, these regulations are ambiguous and open to interpretation. Furthermore,  there exists no transparent and uniform mechanism via which honest victims can  \emph{prove} their innocence. Currently, each bank uses its own ad-hoc (manual) dispute resolution process which is not transparent to customers, regulators, or consumer protection organisations and is not uniform among all banks and even among those organisations that settle disputes between banks and customers. The APP frauds are not specific to the regular online banking systems, they will eventually find their way into other payment systems, such as cryptocurrency.   
%
%
%The two aforementioned issues are not specific to traditional banking. They are applicable to (and could be even worse within) cryptocurrency payment systems as well. To date, there is no official report on the occurrence of APP fraud in cryptocurrency payment platforms. This can be due to the lack of an oversight organisation which collects fraud-related data or due to a low rate of such frauds taking place
%on these platforms because these platforms are not yet popular enough. Nevertheless, with the increase in
%the popularity of cryptocurrency payment systems, it is likely that APP fraudsters will target these platforms'
%users on a large scale as well. 
%
%
%
%Thus, there is a pressing call for effective solutions that can deal with this type of fraud, which is taking victims in the UK and all around the world.  Such an urgent need has been emphasised by the UK's ``Her Majesty's Treasury'' very recently, in May 2022. 
%%
%
%
% APP fraud is a \emph{global} phenomenon. According to the FBI's report, victims of APP frauds reported to it at least a total of  \$$419$ million losses, in 2020 \cite{internet-crime-report}. Recently, Interpol warned its member countries about a variant of APP fraud called investment fraud via dating software \cite{interpol-notce}. According to  Europol’s notice, at least five variants of APP fraud are among the seven most common types of online financial fraud \cite{europol-notice}. 
%
%
% In this project, for the first time, we will propose solutions to deal with this type of fraud. Specifically, we will address the aforementioned two issues on two different payment systems; namely, (i) traditional online banking and (ii) cryptocurrency platforms. Hence, this project will propose solutions that can help (1) decrease the rate of APP fraud occurrence and (2) increase the level of protection that APP victims receive, in both payment systems. To achieve our objectives, we will design cryptographic-based solutions. Cryptography-based solutions have been proven to be effective to combat online adversaries in various contexts. Nevertheless, to date, there have been no cryptographic-based proposals to combat APP fraud. For the first time, we will propose novel cryptographic-based solutions to deal with such fraud. 
 %%%%%%%%%%
 






